% !TEX root = definitions_master.tex

\section{Cryptographic Preliminaries} % (fold)
\label{sec:cryptographic_preliminaries}


\subsection{Cryptographic Tools} % (fold)
\label{sub:cryptographic_defs}

\paragraph{Security parameter.} % (fold)
\label{par:def_security_parameter}

In order to properly formalize security notions, we need to bound the computing power of an attacker.
Indeed, one can always break cryptosystems using a large enough computer and spending a high amount of time.
However, in cryptography, we restrict ourselves to the defence against \emph{reasonable} attackers.
To do so, we use the notion of \emph{security parameter}, denoted $\secpar \in \NN$.
The security parameter is passed as an input to the attacker, under its unary representation $\secparam$, and we only consider attackers whose running time is polynomial in $\secpar$, and whose success probability is non-negligible in $\secpar$.

All these notions are formally defined in the following paragraphs.

% paragraph def_security_parameter (end)

\paragraph{Adversaries.} % (fold)
\label{par:def_adversaries}

An adversary is a probabilistic Turing machine, which, in this manuscript, run in polynomial time, which may carry a state $\mathsf{st}$ when they need to be called several times.
In most cases, we implicitly give as input to the adversary, both the unary representation of the security parameter, and the state.
As a consequence, as adversaries' inputs are always polynomial in the security parameter, the polynomial time adversary runs in time polynomial in the security parameter.

% paragraph def_adversaries (end)


\paragraph{Games.} % (fold)
\label{par:def_games}

Security notions are often defined using security games (or experiments).
Simple games are defined by having an adversary accessing a set of oracles, sometimes with some restrictions on calls to these oracles, and the output of the game is defined as the output of the adversary.

More generally, games are defined using the \emph{code-based games} formalism introduced in~\cite{EC:BelRog06}.
Such a game $G$ is a set of oracle procedures -- including an initialization $\Init$ procedure and a finalization $\Final$ procedure -- that is executed with an adversary $A$, \emph{i.e.} $A$ has access to the procedures, with some possible restrictions. 
For instance, the $\Init$ oracle is always the first one to be called and $\Final$ the last one, once $A$ halted, taking $A$'s output as input. The output of $\Final$ is called the output of the game and is denoted $G^A(\secparam)$. When $\Final$ is omitted, it just forwards the adversary's output. 
		
In those games, at startup, the boolean variables are initialized to $\false$ and the integer variables to $0$.

% paragraph def_games (end)

\paragraph{Statistical Indistinguishability} % (fold)
\label{par:def_stat_ind}

Let $\cD_1$ and $\cD_2$ be two distributions over the set $S$.
$\cD_1$ and $\cD_2$ are said to be \emph{statistically indistinguishable} if 
\[
	\Delta(\cD_1,\cD_2) \leq \negl.
\] 
We denote
\[
	\cD_1 \approx \cD_2
\]
the fact that $\cD_1$ and $\cD_2$ are statistically indistinguishable.


% paragraph def_stat_ind (end)

\paragraph{Computational Indistinguishability} % (fold)
\label{par:def_comp_ind}

Let $\cD_1$ and $\cD_2$ be two distributions which can be sampled in polynomial-time in $\secpar$, and $A$ be a polynomial-time adversary $A$ outputting a single bit. 
The advantage of $A$ distinguishing $\cD_1$ and $\cD_2$ is defined by
\[
	\Adv^{\cD_1, \cD_2}(A,\secparam) = \left| \Prob[A(\cD_1) = 1] - \Prob[A(\cD_2) = 1] \right|.
\]
Distributions $\cD_1$ and $\cD_2$ are said to be \emph{computationally indistinguishable} if for any polynomial-time adversary $A$, the advantage of $A$ in distinguishing $\cD_1$ and $\cD_2$, denoted $\Adv^{\cD_1, \cD_2}(A,\secparam)$, is negligible in $\secpar$.
We denote
\[
	\cD_1 \indist \cD_2
\]
the fact that $\cD_1$ and $\cD_2$ are computationally indistinguishable.
Note that two statistically indistinguishable distributions are computationally indistinguishable.

Similarly, we say that two different games $G_0$ and $G_1$, both outputting one bit, are indistinguishable if, for any polynomial-time adversary $A$, the advantage of $A$ in distinguishing $G_0$ and $G_1$, denoted $\Adv^{G_0, G_1}(A,\secparam)$ and defined as
\[
	\Adv^{G_0, G_1}(A,\secparam) = \left| \Prob[G_0^A = 1] - \Prob[G_1^A = 1] \right|,
\]
is negligible in $\secpar$.
In this case, we write $G_0 \indist G_1$.
% paragraph def_comp_ind (end)

\paragraph{Game-based proofs} % (fold)
\label{par:def_game_based_proofs}

Many of the security notions that we will define and use in this thesis are based on the indistinguishability of two different games $G_0$ and $G_1$.
Unfortunately, in many cases, we will not be able to directly prove this indistinguishability.
Instead, we proceed by \emph{game hops}, by constructing a sequence of games, starting with $G_0$, and ending with $G_1$, and proving that consecutive games are indistinguishable.
The distinguishing advantage between $G_0$ and $G_1$ of an adversary $A$ will then be the sum of the distinguishing advantages of $A$ between every pair of consecutive games in the games sequence. 


% paragraph def_game_based_proofs (end)

\paragraph{The Random Oracle Model (ROM)} % (fold)
\label{par:def_rom}

The Random Oracle Model (or ROM), formally introduced by Bellare and Rogaway in~\cite{CCS:BelRog93}, is a computational model where all parties have access to a (public) random oracle.
As its name indicates, a random oracle outputs a random string for every new input it is given.

To prove the security of some schemes in the ROM, we often use an additional feature, called \emph{programmability}.
This feature allows the games to pre-program the output of the random oracle on some inputs, in a way that the programmed random oracle is indistinguishable from a regular random oracle. 

The ROM is a very useful tool to show the security of some schemes.
However, in practice, random oracles cannot exist (they would require an infinite description), and are often instantiated using hash functions.
There actually is much debate among cryptographers on the quality of the ROM as an abstraction to analyze the security of cryptosystems~\cite{EPRINT:KobMen15}.
Yet, for applied and real-world cryptography, it is a widely accepted and widely used model, as there is no convincing evidence that ROM-protocols have non-theoretical security weaknesses.

% paragraph def_rom (end)

% subsection cryptographic_defs (end)

\subsection{Hardness Assumptions} % (fold)
\label{sub:hardness_assumptions}

Cryptographic primitives rely on the hardness of some mathematical problems.
We describe here the ones that will be useful for our constructions.


\paragraph{The RSA assumption} % (fold)
\label{par:rsa_assumption}

The RSA assumption, as introduced by Rivest, Shamir and Adleman in~\cite{RSA78} 
states that it is infeasible to compute the $e$-th root of an element modulo $N$ when $N$ is a product of two large primes, and $e$ is relatively prime with $\varphi(N)$.

Let $\mathsf{RSAGen}$ be defined as the function, which, on input the security parameter $\secparam$, randomly samples two distinct $\secpar$ bits primes $p$ and $q$, computes $N = p \cdot q$, randomly picks an integer $e$ less than and relatively prime to $\varphi(N) = (p-1)(q-1)$, and outputs the pair $(N,e)$.
For any adversary $A$, let $\advantage{RSA}{A}$ be defined as:
\[
	\advantage{RSA}{A} = \Prob[(N,e) \gets \mathsf{RSAGen}(\secparam), y \getsr \ZZ^*_N, x \gets A(\secparam, N,e,y) : x^e = y \bmod N].
\]
The RSA problem is supposed hard: for any polynomial-time adversary $A$, $\advantage{RSA}{A}$ is negligible in $\secpar$.

% paragraph rsa_assumption (end)

\paragraph{Discrete Logarithm} % (fold)
\label{par:discrete_logarithm}

Solving the discrete logarithm problem in the cyclic group $\GG$ with generator $g$, and of order $N$ consists in finding the integer $x \in \ZZ_N$ such that $g^x = h$ for an element $h \in \GG$.

In terms of security games, this can be formalized as follows.
For any adversary $A$, let $\advantage{DL}{\GG, A}$ be the quantity
\[
	\advantage{DL}{\GG, A} = \Prob[h \getsr \GG, x \gets A(\secparam, \GG, g, h) :  g^x = h ].
\]
We say that the discrete logarithm is hard in $\GG$ if for all polynomial-time adversary $A$, $\advantage{DL}{\GG, A}$ is negligible in $\secpar$.
The discrete log is supposed to be hard on large prime order subgroups of $(\FF_p^*, \times)$, and on cyclic subgroups of elliptic curves over finite fields.

% paragraph discrete_logarithm (end)

\paragraph{The Diffie-Hellman Assumptions} % (fold)
\label{par:def_DH}

A strengthening of the discrete logarithm assumption is the Computational Diffie-Hellman (CDH) assumption.
It requires that an adversary, given $g^a$ and $g^b$, for $g$ a generator of the group $\GG$ of order $N$, and $a,b \in \ZZ_n$, cannot efficiently compute $g^{a \cdot b}$.
Formally, for an adversary $A$, we define the advantage $\advantage{CDH}{\GG, A}$ as
\[
	\advantage{CDH}{\GG, A} = \Prob[a \getsr \ZZ_N, b \getsr \ZZ_N, h \gets A(\secparam, \GG, g^a, g^b) :  g^{a \dot b} = h ].
\]
We say that the CDH assumption is hard in $\GG$ if for all polynomial-time adversary $A$, $\advantage{CDH}{\GG, A}$ is negligible in $\secpar$.
The CDH assumption is supposed to be hard on large prime order subgroups of $(\FF_p^*, \times)$, and on cyclic subgroups of elliptic curves over finite fields.

A stronger assumption is also very commonly encountered, the decisional version of CDH, called the Decisional Diffie-Hellman (DDH) assumption.
This time the adversary is asked to distinguish between the triple $(g^a, g^b, g^{a \cdot b})$ and the triple $(g^a, g^b, g^{c})$.
\[
	\advantage{DDH}{\GG, A} = 
				\begin{multlined}[t]
				\left| \Prob[(a,b) \getsr \ZZ_N^2 : A(\secparam, g^a, g^b, g^{ab}) = 1] \right.\\
				\left. - \Prob[(a,b,z) \getsr \ZZ_N^3 : A(\secparam, g^a, g^b, g^{z}) = 1] \right|.
				\end{multlined}
\]
We say that the DDH assumption is hard in $\GG$ if for all polynomial-time adversary $A$, $\advantage{DDH}{\GG, A}$ is negligible in $\secpar$.
The DDH assumption is also supposed to be hard on large prime order subgroups of $(\FF_p^*, \times)$, and on cyclic subgroups of elliptic curves over finite fields.


% paragraph def_DH (end)

\paragraph{Cryptographic Pairings} % (fold)
\label{par:def_pairings}

We will require that bilinear groups satisfy a hardness assumption called the Decisional Bilinear Diffie-Hellman (DBDH) assumption~\cite{EC:BonBoy04b}.
This assumption requires that a bounded adversary cannot distinguish the tuple $(g, g^a, g^b, g^c, e(g,g)^{abc})$ from the tuple $(g, g^a, g^b, g^c, e(g,g)^z)$, where $a$, $b$, $c$ and $z$ are randomly generated. 

Formally, for a bilinear group $(\GG_1, \GG_2, \GG_T, e)$, the advantage $\advantage{DBDH}{\GG_1, \GG_2, \GG_T, e, A}$ of an adversary $A$ in the Decisional Bilinear Diffie-Hellman game is defined as:
% \begin{align*}
% 	\advantage{BDH}{\GG_1, \GG_2, \GG_T, e, A} =
% 		 		& \left| \Prob[(a,b,c) \getsr \ZZ_N^3 : A(\secparam, g^a, g^b, g^c, e(g,g)^{abc}) = 1] \right.\\
% 				& \tab \left. - \Prob[(a,b,c,z) \getsr \ZZ_N^4 : A(\secparam, g^a, g^b, g^c, e(g,g)^{z}) = 1] \right|.
% \end{align*}
\[
	\advantage{DBDH}{\GG_1, \GG_2, \GG_T, e, A} = 
				\begin{multlined}[t]
				\left| \Prob[(a,b,c) \getsr \ZZ_N^3 : A(\secparam, g^a, g^b, g^c, e(g,g)^{abc}) = 1] \right.\\
				\left. - \Prob[(a,b,c,z) \getsr \ZZ_N^4 : A(\secparam, g^a, g^b, g^c, e(g,g)^{z}) = 1] \right|.
				\end{multlined}
\]
The bilinear group is said to be secure if, for all polynomial-time adversary $A$, $\advantage{DBDH}{\GG_1, \GG_2, \GG_T, e, A}$ is negligible in $\secpar$.


Note that, in this definition, we only considered the symmetric setting for the bilinear group, but that the definition can trivially be adapted to an asymmetric pairing.
In practice, cryptographic pairings are instantiated using elliptic curves, and we will use Type-3 pairings only.
We refer to~\cite{EPRINT:GalPatSma06} for more details on pairings.

% subsection def_pairings (end)


% subsection hardness_assumptions (end)

% section cryptographic_preliminaries (end)
